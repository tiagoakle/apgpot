\documentclass{article}
\usepackage{amsmath}
\newcommand{\pmat}[1]{\begin{pmatrix}#1\end{pmatrix}}
\newcommand{\minres}{ MINRES }
\newcommand{\matlab}{ MATLAB }
\newcommand{\netlib}{ NETLIB }
\newcommand{\n}[1]{\left\| #1 \right\|}
\newcommand{\eps}{\epsilon}
\newcommand{\s}[1]{\operatorname{Span}\left( #1 \right)}
% Commands for problem definition
\newcommand{\maximize}[1]{\underset{#1}{\operatorname{maximize}}\ }
\newcommand{\minimize}[1]{\underset{#1}{\operatorname{minimize}}\ }
\newcommand{\st}{\text{ subject to }}
% Hand made theorem
\newcounter{thm}[section]
\renewcommand{\thethm}{\thesection.\arabic{thm}}
\def\claim#1{\par\medskip\noindent\refstepcounter{thm}\hbox{\bf \arabic{chapter}.\arabic{section}.\arabic{thm}. #1.}
\it\ %\ignorespaces
}
\def\endclaim{
\par\medskip}
\newenvironment{thm}{\claim}{\endclaim}

%Linear algebra notation
\newcommand{\w}{\ensuremath{\mathcal{W}}\ }
\newcommand{\vc}{\ensuremath{\mathcal{V}}\ }

%Shortcuts 
\newcommand{\p}[1]{\ensuremath{\left\( #1 \right\)}}
\newcommand{\tendsto}{\rightarrow}

%Environments
\newtheorem{clm}{Claim}
\newtheorem{prf}{Proof}

%Proximal point shortcuts
\newcommand{\prx}{\operatorname{prox}}
\newcommand{\xp}{x^+}



\title{Potential reduction functions}
\author{Santiago Akle}
\begin{document}
\maketitle
%\section{Background}
%Derived from a reinterpretation of Karmakar\'s algorithm.
\section{Definitions}
Assume that a linear program (LP) is given in standard form
\begin{align*}
  \minimize c^Tx\\
  \st Ax = b,\\
  0 \leq x.
  \label{eq:pprimal}
  \tag{Pp}
\end{align*}
Its dual will be given by \marginpar{\em lagrange dual?} 
\begin{align*}
  \maximize b^Ty\\
  \st A^Ty + z = c,\\
  0 \leq z.
  \label{eq:pdual}
  \tag{Pd}
\end{align*}

\subsection{Properties of the solution}
If both \eqref{eq:pprimal} and \eqref{eq:pdual} are
feasible, then a sufficient and necessary condition 
for a triplet $(x^\star,y^\star,z^\star)$ to be a solution of \eqref{eq:pprimal}
is the satisfaction of the KKT conditions:
\begin{eqnarray*} 
  Ax = b,\\
  A^Ty + z = c,\\
  0\leq x, \\
  0\leq z, \\
  x_iz_i = 0.
  \label{eq:kkt}
\end{eqnarray*}

\subsection{Potential reduction functions}
Denote by $\phi$ the function
\begin{equation*}
  \phi_\rho(x,z) = \rho\log(x^Tz) -\sum_j\log{x_jz_j},
  \label{eq:phi}
\end{equation*}
and by $\psi$ the function
\begin{equation*}
  \psi_\rho(x,y,z) = \frac{\rho}{2}\log\left(\n{Ax-b}^2+\n{A^Ty+z-c}^2 + (c^Tx-b^Ty)^2\right) -\sum_j\log{x_jz_j}
  \label{eq:psi}
\end{equation*}
and, for reasons that will become apparent, impose the condition $\rho >n$.
\subsubsection{Some observations}
\begin{itemize}
  \item If $\rho > n$ then $\phi$ and $\psi$ have no stationary points.
  \item The pair $x,z$ that minimizes $\phi$ subject to $x^Tz = n \mu$ is the point such that $x_iz_i = \mu$.
  \item The triplet that minimizes $\psi$ subject to $x^Tz = n\mu$ to $Ax=b$ and $A^Ty + z - c$ is the point such that $x_jz_j = \mu$. \marginpar{\em verify}
\end{itemize}

\subsubsection{Derivatives}
Though the Hessian of $\psi$ seems to be mildly complicated the gradient has a
relatively simple expression. 
Denote the sum of the square infeasibilities by 
\[R = \n{Ax-b}^2+\n{A^Ty+z-c}^2 + (c^Tx-b^Ty)^2,\]
now the gradient takes the form 
\[
\nabla \psi = \begin{cases}
              \frac{\rho}{R} \left(A^T(Ax-b) + c(c^Tx-b^Ty)\right) - X^{-1}e\\
              \frac{\rho}{R} \left(A(A^Ty+z-c)-b(c^Tx-b^Ty)\right)\\
              \frac{\rho}{R} \left(A^Ty+z-c\right)                              - Z^{-1}e.
              \end{cases}
\]

\subsubsection{Inexistence of stationary points}
A simple argument to show that $\psi$ has no stationary points, begins 
by prooving that for $\rho>n$ $\phi$ has no stationary points.
To do this just 

\subsubsection{Solving LPs with potential reduction functions}
    It is easy to see that if $x_k,y_k,z_k$ is a sequence of bounded iterates and
    $x^\star,y^\star,z^\star$ is a limit point such that $\psi(x_k,y_k,z_k) \rightarrow - \infty$
    as $(x_k,y_k,z_k) \rightarrow (x^\star,y^\star,z^\star)$, then the limit point is a solution
    of \eqref{eq:pprimal}.

    From the boundedness of the iterates we can conclude that the second term is bounded from below,
    therefore a reduction of $\psi$ or $\phi$ to minus infinity implies that the 
    first term must tend to minus infinity, and this can only happen if its argument tends to zero.

\subsubsection{Relation to the central path}
    If both the primal and dual are stictly feasible the central path exists. 
    Denote by $x(\mu),y(\mu),z(\mu)$ the point in the central path parametrized by $\mu$ and 
    observe that 
    \[
    \psi(x(\mu),y(\mu),z(\mu)) = \phi(x(\mu),y(\mu),x(\mu)) = \rho\log(n\mu) - n\log\mu = \rho\log n + (\rho-n)\log \mu. 
    \]
    Clearly if $\rho >n$ as $\mu\tendsto0$ $\psi \tendsto-\infty$.
    In other words if we follow the central path to the solution the functions tend 
    to $-\infty$.

\section{Simplified Homogeneous self dual formulation} 

The homogeneous self dual formulation of \cite{SimplifiedHSDYe}, embeds the LO into a larger one (HLO).
This LO is its own dual and is allways feasible. A consequence of this is that there is allways a solution of 
this larger LO. 
\begin{align}
  \min 0 \\
  \st \\
  Ax - \tau b = 0\\
  -A^Ty +\tau c \geq 0\\
  c^Tx-b^Ty \geq 0\\
  x\geq 0\\
  \tau \geq 0
  \tag{HSO}
  \label{eq:hsd}
\end{align}

Denote by $z$ the slacks for the second equation and by $\kappa$ the slack scalar for the dual gap.
The problem is feasible since $0$ is a feasible solution. Furthermore if a solution is strictly complementary
we have that $\tau + \kappa >0$ and $\tau \kappa = 0$. If in this case $\kappa >0$ then the original lp
is infeasible, if however $\tau>0$ the solution is easily calculated.

\section{Potential reduction for the HSD}
\begin{equation*}
  \psi_H(x,y,z) = \frac{\rho}{2}\log\left(\n{Ax-\tau b}^2+\n{A^Ty+z-\tau c}^2 + (c^Tx-b^Ty +\kappa)^2\right) -\sum_j\log{x_jz_j} - \log\left( \tau \kappa \right)
  \label{eq:psiH}
\end{equation*}

\subsection{Derivatives}
Assume the variables are ordered as $x,\tau,y,z,\kappa$ then the gradient takes the form:
\[
\nabla \psi_H = \begin{cases}
              \frac{2\rho}{R} \left(A^T(Ax-\tau b) + c(c^Tx-b^Ty + \kappa)\right) - X^{-1}e\\
              \frac{2\rho}{R} \left( b^T(\tau b-Ax)+c^T(\tau c - z- A^Ty) \right)  - \tau^{-1}\\
              \frac{2\rho}{R} \left(A(A^Ty+z-\tau c)-b(c^Tx-b^Ty + \kappa)\right)\\
              \frac{2\rho}{R} \left(A^Ty+z-\tau c\right)                              - Z^{-1}e\\
              \frac{2\rho}{R} \left(c^Tx-b^Ty+\kappa \right)  - \kappa^{-1}.
              \end{cases}
\]

Denote by 
\begin{align}
  p = Ax-\tau b,\\
  d = A^Ty + z - \tau c,\\
  g = c^Tx-b^Ty+\kappa.
  \label{eq:defres}
\end{align}
Then the gradient can be written as:
\[
\nabla \psi_H = \begin{cases}
              \frac{2\rho}{R} \left(A^Tp + gc\right) - X^{-1}e\\
              \frac{2\rho}{R} \left( -b^Tp-c^Td \right)  - \tau^{-1}\\
              \frac{2\rho}{R} \left(Ad-gb\right)\\
              \frac{2\rho}{R} d                               - Z^{-1}e\\
              \frac{2\rho}{R} g  - \kappa^{-1}.
              \end{cases}
\]



\section{Convergence rates}

\section{Achieving sufficient descent}
Denote by 
\[
E = \pmat{ & A^T & I \\
         A &     &   \\
         c^T&-b^T}
\] 
the $m+n+1$ by $2n+m$ system, by $s_k = (x_k,y_k,z_k)$ the $k$th iterate,
by
\[
r = -\pmat{d_r,p_r,g_r}
\] the residual and by 
$\iota(s_k) = \n{E s_k-r}$, the norm of the infeasibility.

If at iteration $k$ the search direction is chosen so that $E\Delta s = -r$, then
a damped Newton step of the form $s_{k+1} = s_k + \alpha \Delta s$ for $\alpha <1$ 
will achieve a reduction in the magnitude of the residuals of the form 
\[
(1-\alpha)(Es_k-r) = (Es_{k+1}-r),
\]
which in turn implies that $\iota(s_{k+1})/\iota(s_k) = 1-\alpha$. 

Observe that 
\[ \psi(s_{k+1}) - \psi(s_k) =
\rho \log\left( \iota^2(s_{k+1})/\iota^2(s_k) \right) 
- \sum\left( \log\left( 1-\frac{\Delta x_j}{x_j} \right) 
+ \log\left(1-\frac{\Delta z_j}{z_j} \right) \right)
\]
which implies that 
\[
\psi(s_{k+1}) - \psi(s_k) = 
\rho \log\left( (1-\alpha)^2 \right) 
- \sum_j\left( \log\left( 1-\frac{\alpha \Delta x_j}{x_j} \right) 
+ \log\left(1-\frac{\alpha \Delta z_j}{z_j} \right) \right).
\]
Assume that the step length $\alpha$ is chosen such that $\n{\alpha X^{-1}\Delta x}_\infty < \tau < 1$, 
and $\alpha < \tau$.

Using the bound \cite{Kojima}
\[
\sum_j \log(1-x_j) \geq -e^Tx-\frac{\n{x}^2}{2(1-\tau)}
\]
which is valid for $\n{x}_\infty\leq1$,  \marginpar{\em Cite Mizuno} 
we can bound the difference by 
\[ 
\Delta \psi = \rho \log\left( 1-\alpha \right)^2 + \alpha e^T(X^{-1}\Delta x
+ Z^{-1}\Delta z) + \alpha^2\left( \frac{\n{Z^{-1}\Delta z}_2^2+\n{X^{-1}\Delta
x}_2^2}{2(1-\tau)} \right).
\]

\begin{eqnarray*}
  \Delta \psi \leq& \rho \log \left( (1-\alpha)^2 \right) + \alpha e^T(X^{-1}\Delta x + Z^{-1}\Delta z) + \frac{\alpha^2}{2(1-\tau)} 
  \pmat{\Delta x & \Delta y} \pmat{X^{-2} & \\ & Z^{-2}}\pmat{\Delta x \\ \Delta z} \\
  \Delta \psi \leq& \rho \log \left( (1-\alpha)^2 \right) + Q(\Delta x, \Delta z).
\end{eqnarray*}

\section{An Infeasible Potential Reduction Function?}

\[
\psi(x,y,z) = \rho \log\left\{ \frac{\n{Ax-b}^2 + \n{A^Ty+z-c}^2 + (c^Tx-b^Ty)^2}{x^Tz} \right\} - \sum_j\log{x_jz_j}
\]

Observe that if $s = (x,y,z)$ is in the central path, then:
\[
\psi(s) = \rho \log\left\{ (x^Tz)^2/x^Tz \right\} - \sum_j \log{x_jz_j} = \phi(s)
\]
and will tend to $-\infty$ as $\mu$ tends to zero.

Assume that the solution set is bounded, and assume that $s_j$ is a bouned
sequence with an accumulation point $s^\star$ such that for some sub sequence
$s_{j_k}$ we have $\psi(s_{j_k})\rightarrow -\infty $ as $k\rightarrow \infty$.

Then from the boundedness of the sequence we have that the term $\sum\log{x_jz_j}$ is 
bounded bellow. Therefore convergence of $\psi{s_k}$ to $-infty$ implies
that the argument of the first term tends to zero.

This in turn implies that the numerator tends to zero proving the convergence of $s_k$ to a solution.
\end{document}
